% Regular Languages TODO
% - NFA formalism
% - Talk about NFA/DFA conversion
% - NFA/DFA closures
%   e.g. answer 'how do I reverse an NFA
% - Pumping lemma shortcuts
% -! DFA minimization
%   - Brzozowski’s algorithm
%   - Table-construction algorithm
% - DFA to RE-conversion
A regular language is any language that can be recognized with a DFA.
Formally a DFA is a tuple $(Q,\Sigma,\delta,q_0,F)$. Where:

\begin{tabular}{rp{0.79\linewidth}}
$Q$             & A finite, non-empty set of states. \\
$\Sigma$        & A finite, non-empty alphabet. \\
$\delta$        & A function ($\delta : Q \times \Sigma \to Q$) that maps
                  a state, and an input in $\Sigma$ to a new state. \\
$q_0$           & A state in $Q$ that DFA starts execution from. \\
$F \subseteq Q$ & A finite, possibly empty, set of accepting states.
\end{tabular}
Alternatively, regular languages can be defined by an NFA. Formally, NFAs are
the same as DFAs, except the $\delta$ function for NFAs is defined as:
\[
    \delta : Q \times (\Sigma \cup \{\varepsilon\}) \spto 2^Q
\]
Where $2^Q$ represents the power-set of $Q$. Basically, the delta function can
now map an input to multiple states instead of just one state.

\subsection{Closures}
Where $R$ is a regular language, $L$ is `not regular', and $?$ is Unknown.

% Figure out the width of the third coulumn
\settowidth{\templength}{$R \cap R \spto R$}
\addtolength{\templength}{1cm}
\begin{tabular}{lp{\templength}l}
\textbf{Closed:}       &                    & \textbf{Unclosed:} \\
$\overline{R} \spto R$ & $h(R) \spto R$     & $R \cap L \spto ?$ \\
$R^* \spto R$          & $R \cup R \spto R$ & $R \cup L \spto ?$\\
$R^R \spto R$          & $R \cap R \spto R$ & $L \cup L \spto ?$\\
$RR \spto R$           & $R \spbackslash R \spto R$ & \\
\end{tabular}

\subsection{Pumping Lemma}
\begin{align*}
\exists N \in \mathbb{N}: & \\
         \forall w \in L: &\; |w| \geq N \impl \\
\exists xyz \in \Sigma^*: & \quad w = xyz \\
                          & \land |xy| \leq N \\
                          & \land y \neq \varepsilon \\
                          & \land \forall i \geq 0: xy^iz \in L
\end{align*}
